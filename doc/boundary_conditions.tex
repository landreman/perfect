\documentclass[12pt]{article}
\usepackage[a4paper,margin=2cm]{geometry}
\usepackage{amsmath}

\title{Boundary Conditions in PERFECT}
\date{\today}

\begin{document}
\maketitle

This document refers to the Fortran version of PERFECT.

PERFECT solves a differential equation which is first order in radial
derivatives. It therefore requires one radial boundary condition for each
$(x,\xi,\theta)$ point, while there are two radial boundaries. In order to
correctly specify the system, the boundary conditions are fixed at the poloidal
positions on the boundaries where particle trajectories {\em enter} the domain,
i.e.~where $\boldsymbol{v}_\mathrm{d}.\nabla\psi>0$ on the left boundary or
$\boldsymbol{v}_\mathrm{d}.\nabla\psi<0$ on the right boundary.

The boundary conditions in PERFECT are set by the options {\tt
leftBoundaryScheme} (for {\tt ipsi=1}) and {\tt rightBoundaryScheme}
(for {\tt ipsi=Npsi}). These are integers and may have values {\tt 0}, {\tt 1}
or {\tt 2}, whose meanings are described below.


\section*{{\tt 0}: Dirichlet to Maxwellian}

The perturbed distribution function, $\hat{g}_a$, is set to zero. Therefore the
distribution function is just the Maxwellian part $f=f_\mathrm{M}(\psi)$.

\section*{{\tt 1}: Dirichlet to local solution}

This assumes that the radial gradients are small at the boundaries, so that the
solution to the local drift kinetic equation is a good approximation to the
solution to the global drift kinetic equation.

First the local solution at {\tt ipsi=1} (for the left boundary) or {\tt
ipsi=Npsi} (for the left boundary) is computed. Then the part of $\hat{g}$ in
the null space of the drift kinetic operator (i.e. the parts of the density and
pressure moments that are constant in $\theta$) is subtracted from the
solution, so that it satisfies the constraints $\left< \int d^3v\, \hat{g}_a
\right> = 0$ and $\left< \int d^3v\, x_a^2 \hat{g}_a \right> = 0$. Finally the
solution to the local drift kinetic equation thus obtained is used as a
Dirichlet boundary condition for the perturbed distribution function,
$\hat{g}_a$, in the global solution.

\section*{{\tt 2}: Neumann boundary conditions}

This again assumes that the global solution approaches the local one at the
boundaries, but in a slightly less restrictive way.

The global terms in the drift kinetic equation (i.e.~those which are switched
off when {\tt makeLocalApproximation=true}) are dropped from the matrix for the
global drift kinetic operator on the points where the boundary condition is to
be applied, while the local terms (which remain when {\tt
makeLocalApproximation=true}) are retained. It has been called a Neumann
boundary condition since the terms that are dropped include the radial
derivatives, so that the normal derivative is set to zero.

In contrast to option {\tt 1}, in this case the `local' part of the solution (on
one half of the boundary flux surface) is able to respond to the `global' part
(on the other half of the boundary flux surface), to which it is coupled by the
poloidal derivatives. There is therefore less inconsistency between the points
on the flux surface on which the boundary condition is applied and those on
which it is not; this appears to be numerically beneficial (the condition number
of the matrix is reduced, at least in some cases).

\end{document}
